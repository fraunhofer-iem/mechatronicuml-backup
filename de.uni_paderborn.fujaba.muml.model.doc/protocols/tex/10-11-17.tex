%%This is a very basic article template.
%%There is just one section and two subsections.
\documentclass[11pt,a4paper]{article}
\usepackage[ngerman,german]{babel}
\usepackage[utf8]{inputenc}
\usepackage[T1]{fontenc}
\usepackage[left=3.5cm,right=3.5cm,top=3cm,bottom=3cm]{geometry}

\begin{document}

\begin{center}

\textbf{\huge Protokoll SHK-Treffen 17.11.10}\\[0.9cm]

\end{center}

\begin{itemize}
  \item GMF-Pattern-Editor
  \begin{itemize}
    \item Namens-Label soll horizontal und vertikal im Pattern zentriert werden.
    \item Nachsehen ob GMF eine generische Diagramm-Klasse anbietet.
    \item Namen der Rollen müssen noch hinzugefügt werden.
  \end{itemize}
  \item \emph{instance}-Package
  \begin{itemize}
    \item Alle Klassen sollen mit \emph{Instance} gesuffixt werden.
    \item Der Package-Name bleibt erstmal \emph{instance} bis das mit Claudia
    diskutiert ist.
    \item \emph{PortInstance} wird evtl. abgeschafft, da Portinstanzen ebenso
    wie Portparts anhand des Komponententypen herleitbar sind.
    \item Denis macht bis zum nächsten Mal ein Paar Beispiel für
    Deployment-Diagramme.
    \item Bzgl. des \emph{instance}-Package ist noch einiges an Diskussion
    fällig.
  \end{itemize}
  \item Constraints
    \begin{itemize}
      \item Constraints werden entsprechend der Klassenhierarchie umgesetzt, wie
      sie Christian skizziert hat.
      \item In Zukunft gibt es Modeling Constraints und Verifiable Constraints.
      \item Elemente an denen Constraints verifiziert werden können erben von
      der neuen Klasse \emph{ConstrainableElement}.
    \end{itemize}
  \item Klasse \emph{RealtimeStatechart}
    \begin{itemize}
      \item s weg bei outgoingTransitionss.
      \item \emph{getLongName()} abschaffen, es sei denn sie liefert eine
      sinnvolle qualifizierte Version des Namens.
      \item \emph{getRootRealtimeStatechart()} überprüfen, falls keine Nutzung,
      abschaffen.
      \item Recherchieren, wie die UML hierarchische Zustände in
      Statecharts umsetzt.
    \end{itemize}
    \item Dem RTSC-MM müssen noch deep und shallow History States hinzugefügt
    werden.
    \item Nachsehen ob UML fork- und join-Knoten unterstützt.
    \item \emph{UMLComplexRealtimeState} soll evtl. umbenannt werden.
    \item Ein Internal Event ist kein Event sondern eine Action/ein Seiteneffekt
    und sollte in die entsprechende Klassenhierarchie verschoben werden. Evtl.
    sollte er umbenannt werden in \emph{UMLRealtimeSideEffect}.
    \item Das String-Attribut bei Actions wird abgeschafft. Stattdessen wird
    eine Assoc auf \emph{Expression} hinzugefügt.
    \item Das Attribut \emph{actionType} wird abgeschafft. Der Typ (entry, do,
    exit oder sideeffect) ist implizit durch die Assoc gegeben, die auf die
    Action zeigt.
    \item UMLRealtimeAction.blocking muss vom Typ boolean sein (Anmerkung vom
    Verfasser diese Protokolls ;) : Hab' das im Code recherchiert und die
    WCET-Analyse in
    de.uni\_paderborn.fujaba.umlrt.model.realtimestatechart.algorithms.wcet)
    erwartet tatsächlich einen long-Wert. Das sollte also erstmal so bleiben.)
    \item Aufgabenverteilung:
    \begin{itemize}
      \item Ingo: Bastelt weiter am GMF-Editor.
      \item Julian: RTSC-MM aufräumen und ausmisten, insbesondere die zeitlichen
      Elemente (\ldots WithLowerBound \ldots).
    \end{itemize}
\end{itemize}

\end{document}
