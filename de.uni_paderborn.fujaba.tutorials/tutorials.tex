%\documentclass[12pt,a4paper,twoside,titlepage,headsepline,pointlessnumbers,liststotoc,idxtotoc,bibtotoc,ngerman]{scrartcl}
\documentclass[12pt,a4paper,twoside,titlepage,headsepline,pointlessnumbers,liststotoc,idxtotoc,bibtotoc,ngerman,english]{scrartcl}

\ifx\pdfoutput\undefined
\pdffalse % we are not running pdflatex
\else
\pdfoutput=1 % we are running pdflatex
\pdfcompresslevel=9     % compression level for text and image;
\usepackage[pdftex,bookmarks=true,bookmarksopen=false,bookmarksnumbered=true,linktocpage,colorlinks=true,backref,pagebackref, linkcolor=black,  citecolor=black, urlcolor=black]{hyperref}
\usepackage[pdftex]{graphicx}
\usepackage[final]{pdfpages}

% Setzt die Einrücktiefe der ersten Zeile nach einem Absatz auf 0
\setlength{\parindent}{0pt} 
\usepackage{needspace}


\usepackage{amsfonts}
\usepackage{algorithmic}
\usepackage{bibgerm}
\usepackage{url}
\usepackage{color}
\usepackage{tabularx}
\usepackage[T1]{fontenc} 
\usepackage[utf8]{inputenc}
\usepackage{bbm}
\usepackage{amsmath}
\usepackage{amssymb}
\usepackage{ntheorem}
\usepackage{fancyvrb}
\usepackage{subfigure}
%\usepackage{ngerman}
\usepackage[english]{babel}
\usepackage{mathptmx} 
\usepackage[scaled=.90]{helvet} 
\usepackage{courier}
\usepackage[rightcaption]{sidecap}
\usepackage{lscape}
\usepackage{supertabular} 
\usepackage{graphicx}
\usepackage{tabularx}

% Listings
\usepackage{listings}
\lstset{basicstyle=\small,captionpos=b}

\newenvironment{packed_enum}{
\begin{enumerate}
  \setlength{\itemsep}{0.5pt}
  \setlength{\parskip}{0pt}
  \setlength{\parsep}{0pt}
}{\end{enumerate}}

\RequirePackage{xspace}

\setcounter{secnumdepth}{3}
\selectlanguage{english} % Dokumentensprache auf englisch stellen
%\selectlanguage{german} % Dokumentensprache auf deutsch stellen


\newcommand{\p}[1]{\texttt{\small #1}}
\newcommand{\INFO}[1]{{\small~\\\hrule\vspace{0.1cm}\hrule~\\\textbf{Information:}\\#1~\\\hrule\vspace{0.1cm}\hrule~\\}}
%\theoremstyle{break}
%\newtheorem{defi}{Definition}[section] 
%\newtheorem{bsp}{Beispiel}[section] 

\definecolor{gray}{gray}{0.95}
\setlength{\fboxrule}{2pt}

\providecommand{\ext}[1]{}
\providecommand{\TODO}[1]{{\small ~\\\hrule\vspace{0.1cm}\hrule~\\\textbf{TODO}:  #1~\\\hrule\vspace{0.1cm}\hrule~\\}}

\pagestyle{headings}

\begin{document}                
             

	% ************************************************************
	% *****                    Titelseite                    *****
	% ************************************************************


	\setcounter{page}{1}
	\pagenumbering{Roman}

	\begin{titlepage}
	\thispagestyle{empty}
	\begin{center}

			\includegraphics[width=12cm]{figures/Logo_Uni_Paderborn}\\
			\textsf{
			Fakultät für Elektrotechnik, Informatik und Mathematik \\
			Institut für Informatik, Fachgebiet Softwaretechnik \\ 
      		Warburger Straße 100 \\
			33098 Paderborn} \\
                    
      \vspace{4cm}									

			{\LARGE  \textbf{GMF-Tutorials}} \\ 
			\vspace{0.5cm}
			{\Large Fujaba Diagram Editors with GMF} \\ 
			\vspace{2.5cm}

			\vspace{0.5cm}
			\textbf{Author:} Ingo Budde \\
			\vspace{1cm}
			Paderborn, \today \\
			\vspace{1cm}
			
	\end{center}
	\end{titlepage}

	\clearpage
		
	\thispagestyle{empty}
	
	\tableofcontents
	
	\clearpage
  
  \pagenumbering{arabic}
	\setcounter{page}{1}
	
\section{Overview}
\subsection{Introduction}
In this document several solutions will be provided that help developing a
GMF-Diagram Editor for Fujaba.
It is not necessary to do it exactly this way, but it should give the reader a
good orientation.

\subsection{Multiple ways of modifying the generated editor}
Basically there are multiple ways to modify the look and behavior of the
generated editor.
\begin{enumerate}
    \item Adjusting the generated code directly (myeditor.diagram)
	\item Creation of a "Extension"-Plugin (myeditor.diagram.custom)
\end{enumerate}

In this document, the second variant will be chosen, so that the GMF-Code can be
regenerated without problems. Furthermore, this way there is a strict separation
of generated and manually written sourcecode. This idea was taken from Dr. Köhnlein's
\href{http://www.slideshare.net/itemis/gmf-fr-anspruchsvolle-presentation}{slides}
at w-jax08 (slides 20-22). \\
In case, changes on the generated code are unavoidable, this can be achieved by
modifying the gmf-templates used to generate the code.


\subsection{Conventions}
It is assumed that the editor is called MyEditor. 

\section {Creating a NewWizard for a Fujaba Diagram Editor}
Fujaba Diagram Editor share a domain model file (the fujaba model).
Therefore a New Wizard for Fujaba Diagrams must be able to put a new
diagram into an existing model file, without deleting all of its contents.
An abstract implementation for such a New Wizard exists within the
$de.uni\_paderborn.fujaba.newwizard$ plugin.

\subsection {MANIFEST.MF}
In order to use the abstract implementation, make sure that your
$myeditor.custom$ plugin has a dependency to
$de.uni\_paderborn.fujaba.newwizard$.

\subsection {plugin.xml}
\lstinputlisting
[caption={An extension must be added to the
plugin.xml}\label{lst:javaclass},language=XML]
{sections/newwizard/plugin.xml}


\subsection {Class CustomMyEditorDiagramCreationWizard}
\lstinputlisting
[caption={Example for
myeditor.diagram.custom.part.MyEditorDiagramCreationWizard.}
\label{lst:javaclass},language=JAVA] {sections/newwizard/newwizard.java}


%	\listoffigures % Abbildungsverzeichnis

	% ************************************************************
	% *****                      Ende                        *****
	% ************************************************************

\end{document}