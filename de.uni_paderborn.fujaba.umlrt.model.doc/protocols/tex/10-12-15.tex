%%This is a very basic article template.
%%There is just one section and two subsections.
\documentclass[11pt,a4paper]{article}
\usepackage[ngerman,german]{babel}
\usepackage[utf8]{inputenc}
\usepackage[T1]{fontenc}
\usepackage[left=3.5cm,right=3.5cm,top=3cm,bottom=3cm]{geometry}
\usepackage{graphicx}

%% Add pix with one line.
\newcommand{\inclpic}[5]{\begin{figure}[#1]\begin{center}\includegraphics[width=#2\linewidth]{#3}\caption{#4}\label{pic:#5}\end{center}\end{figure}}

\begin{document}

\begin{center}

\textbf{\huge Protokoll SHK-Treffen 15.12.2010}\\[0.9cm]

\end{center}

\begin{itemize}
  \item Die Action eines RTSCs soll in Zukunft ein Objekt Klassen
  \emph{MethodCallExpression}, \emph{VariableAssignmentExpression} oder
  \emph{TextualExpression} aus dem SDM-MM sein können.
  \item Ein Guard eines RTSCs soll repräsentiert werden durch eine
  \emph{NotExpression}, eine \emph{BinaryExpression} oder eine
  \emph{ComparisonExpression}.
  \item Die Klasse \emph{ParamChannel} wird gelöscht. Stattdessen soll es in
  Zukunft ermöglicht werden, dass ein Statechart, dass sich in der
  Empfängerrolle befindet, bestimmte Werte für die Parameter eines Kanals
  erwarten kann. (Das entspricht sowohl der Semantik von Uppaal als auch der von
  Spin.) Bsp.: Es sei $s(int\;i)$ ein Synchronisationskanal mit einem Parameter
  $i$ vom Typ $int$. Ein Statechart in der Empfängerrolle, dass sich nur genau
  dann synchronisieren soll, falls $i$ den Wert $5$ besitzt, hat eine
  Transition, die mit $s(5)?$ gelabelt ist.
\end{itemize}

\end{document}
