%%This is a very basic article template.
%%There is just one section and two subsections.
\documentclass[11pt,a4paper]{article}
\usepackage[ngerman,german]{babel}
\usepackage[ansinew]{inputenc}
\usepackage[T1]{fontenc}
\usepackage[left=3.5cm,right=3.5cm,top=3cm,bottom=3cm]{geometry}
\usepackage{graphicx}

%% Add pix with one line.
\newcommand{\inclpic}[5]{\begin{figure}[#1]\begin{center}\includegraphics[width=#2\linewidth]{#3}\caption{#4}\label{pic:#5}\end{center}\end{figure}}

\begin{document}

\begin{center}

\textbf{\huge Protokoll SHK-Treffen 10.05.2011}\\[0.9cm]

\end{center}

\begin{itemize}
  \item Realtime Statechart Editor: 
  	\begin{itemize}
					\item relative Deadlines und Priorit�ten k�nnen �ber Propterty View der Transitionen angelegt werden
					\item Guardexpression bisher nur als String angelegt; Expression Parser fehlt noch
					\item Safety Transition: Schaltet nur, wenn Risiko beim Schalten nicht zu hoch ist
					\begin{itemize}
						\item maximal zul�ssiger Risiko-Wert muss f�r das Realtime Statechart angegeben werden
						\item Risikoanalyse liefert Risiko-Wert f�r einzelne Safety Transition
					\end{itemize}
					\item Synchronisationchannels in Zust�nden deklarieren
					\begin{itemize}
						\item Parametertypen k�nnen bisher selbst definiert werden; EDataType soll verwendet werden
						\item werden �ber Kontextmen� der Transitionen instanziiert werden (Visualisierung fehlt)
					\end{itemize}
					\item Clock Constraint Operatoren sollen f�r Invarianten noch auf LESS und LESSEQUAL eingeschr�nkt werden
					\item Entry/Do/Exit-Methoden besitzen Action -> Expression; werden als String eingegeben und geparst
					\item noch nicht angelegt werden k�nnen: Events, absolute Deadlines
		\end{itemize}
	\item Component Instance Editor:
		\begin{itemize}
			\item Delegations anlegen nicht m�glich, da dies beim Anlegen der Komponententypen gemacht wird (Structured Component Editor)
			\begin{itemize}
				\item Nach Instanziierung von Structured Components ist der Typ der inneren Components noch �nderbar, was noch ge�ndert werden soll
				\item Message Interfaces k�nnen in mehreren Diagrammen oder einem einzelnen definiert werden (Nutzerentscheid)
			\end{itemize}
		\end{itemize}	
	\item Bugtracker wird im Geforge-System eingerichtet und soll in Zukunft genutzt werden (Bugs / Feature Requests dort eintragen)	
\end{itemize}

\end{document}
