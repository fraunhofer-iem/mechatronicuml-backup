%%This is a very basic article template.
%%There is just one section and two subsections.
\documentclass[11pt,a4paper]{article}
\usepackage[ngerman,german]{babel}
\usepackage[utf8]{inputenc}
\usepackage[T1]{fontenc}
\usepackage[left=3.5cm,right=3.5cm,top=3cm,bottom=3cm]{geometry}

\begin{document}

\begin{center}

\textbf{\huge Protokoll SHK-Treffen 10.11.10}\\[0.9cm]

\end{center}

\begin{itemize}
  \item Das Meta-Modell für Konnektoren wird umgesetzt wie von Christian an der
  Tafel skizziert.
  \item Alle Elemente mit Namen erben von ``NamedElement''.
  \item Die Klasse ``Role'' erhält zwei Assoziation ``required'' und
  ``provided'' zur Klasse ``MessageInterface''. Vorhandene Assocs zu
  ``MessageInterface'' werden gelöscht.
  \item ``Channel'' erbt von ``BehavioralConnector'', die Assoc ``Channel ->
  UMLRealtimeStatechart'' wird gelöscht.
  \item Für alle Modellelemente derern Verhalten durch ein Realtime Statechart
  beschrieben wird wird eine Basisklasse ``BehavioralElement'' eingeführt.
  \item In der Klasse ``Constraint'' wird die Klasse ``TextualExpression''
  verwendet, die es ermöglicht zugleich einen Ausdruck und die Sprache in der
  der Ausdruck formuliert ist, zu speichern. ``language''-Attribut in
  ``Constraint'' entfällt.
  \item Welche Klassen im Package ``instance'' bestehen bleiben muss mit Claudia
  diskutiert werden.
  \item Julian überlegt sich, wie (parametrisierte) Synchronisationskanäle
  umgesetzt werden können.
\end{itemize}

\end{document}
