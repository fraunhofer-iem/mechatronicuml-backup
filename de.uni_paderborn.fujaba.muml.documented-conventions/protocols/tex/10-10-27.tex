%%This is a very basic article template.
%%There is just one section and two subsections.
\documentclass[11pt,a4paper]{article}
\usepackage[ngerman,german]{babel}
\usepackage[utf8]{inputenc}
\usepackage[T1]{fontenc}
\usepackage[left=3.5cm,right=3.5cm,top=3cm,bottom=3cm]{geometry}

\begin{document}

\begin{center}

\textbf{\huge Protokoll SHK-Treffen 27.10.10}\\[0.9cm]

\end{center}

\begin{itemize}
\item  'ASGElementRef'-Äquivalent im neuen MM heißt 'Extension'
\item  Assoc zwischen Port und Component wird Containment-Beziehung
\item  'PortRole' im Package 'pattern' wird zu 'Role' umbenannt.
\item  Es wird in Zukunft Structured und Atomic Components geben.
\item  Structured Components besitzen kein Statechart, Atomic Components
besitzen Statecharts.
\item  Interface werden nicht mehr an den Ports dargestellt.
\item  Ports erhalten graphischen Repräsentation für die Richtung in From von
 Pfeilen.
\item  Typen und Parts sollten in der Doku kommentiert werden.
\item  Das Komponenten-MM wird wie an der Tafel skizziert umgesetzt.
\item  Für die Umsetzung von PortParts werden Phantom-Nodes verwendet.
\item  Die Assoc von Component zu EClass bleibt, Component erbt nicht von
TypedElement.
\item  Aufgabenverteilung:
\begin{itemize}
\item Julian: Kommentiert das MM.
\item Ingo: Arbeitet sich in GMF ein.
\end{itemize}
\item  Zu GMF gibt es eine gutes Tutorial dessen Beispiel aus Schiffen und
Routen besteht.
\item  'UMLRealtimeStatechart' soll von 'NamedElement' und
'CommentableElement' erben.
\item  'getTopStatechart()' fliegt raus. Statdessen 'isEmbedded()'. Wird als
derived umgesetzt in Abhängigkeit davon, ob eine Parent-SC existiert.
\item  'checkSyntax()' fliegt raus.
\item  Für die qualifiziert Assoc von 'UMLRealtimeStatechart' zu 'UMLClock'
überprüfen, ob diese tatsächlich qualifiziert sein muss. Falls nein,
durch einfache Assoc austauschen.
\item  'observationMap' fliegt raus.
\item  Für 'getFirstCalc()' und 'getRefinement()' betrachten und entscheiden,
ob diese Methoden sind, die im MM untergebracht werden sollten.
\item  Die Assocs von 'UMLRealtimeStatechart' zu 'UMLRealtimeState' und zu
'UMLRealtimeTransition' werden zu Containment-Beziehungen. Allgemein
werden alle Assocs von 'UMLRealtimeState'/'UMLRealtimeTransition' zu
Elementen die zu Zuständen bzw. Transitionen gehören als
Containtment-Beziehung umgesetzt.
\item  Die verschiedenen Actions im RTSC-MM werden durch Enum-Typen umgesetzt.
\item  Die momentan noch seperaten Ecore-Dateien pro Package sollen in einer
Ecore-Datei vereinigt werden.


\end{itemize}

\end{document}
