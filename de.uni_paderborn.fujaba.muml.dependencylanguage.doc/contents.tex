
		\newcommand{\todoo}[1]{\todo[inline, color=red!40]{ecore2latex: #1}}
		
		\section{EPackage \bfseries \texttt{dependencylanguage}\normalfont}
		
		% Here comes the package documentation
		\begin{longdescription}
		\item[Overview]			
				

	

		This package provides a the possiblity to describe dependencies between multiple RTCSs.
The dependencies can be appended to AtomicComponent through the SynthesizableBehavior.

@author Sebastian Goschin		
		\end{longdescription}
	% Here a manual modifiable file is included: dependencylanguage/graphics.tex
	%
% This file has been generated by Ecore to LaTeX written in MWE Xpand
% It is save to alter this file as it WILL NOT be overwritten.
% The file is included by the main latex file in the appropriate place, not further
% actions are required
%
~\missingfigure{Package Diagram missing}

%%%%%%%%%%%%%%%%%%%%%%%%%%%%%%
%%%%%%%%%%%%%%%%%%%%%%%%%%%%%%
%%%%%%%%%%%%%%%%%%%%%%%%%%%%%%
\subsection{EEnumeration \bfseries \texttt{AssignmentDirection}\normalfont}
\label{cls:dependencylanguage::AssignmentDirection} \index{dependencylanguage!AssignmentDirection}

	\begin{longdescription}
		\item[Overview] 		
				

	

		If a DataAssignmentEffect is specified for an Event and Variable of the same region and only reads from others, its called PULL.
Otherwise, it is called PUSH, because the Variable is modified outside of the original owner.		
	
		\item[\textbf{ELiterals of} \texttt{AssignmentDirection}] ~
		\begin{longdescription}
			
\item[\texttt{PUSH = 0}] ~
\nopagebreak

\item[\texttt{PULL = 1}] ~
\nopagebreak
		\end{longdescription}
	\end{longdescription}
	
	

%%%%%%%%%%%%%%%%%%%%%%%%%%%%%%
%%%%%%%%%%%%%%%%%%%%%%%%%%%%%%
%%%%%%%%%%%%%%%%%%%%%%%%%%%%%%
\subsection{EClass \bfseries \texttt{AuxiliaryClockCondition}\normalfont}
\label{cls:dependencylanguage::AuxiliaryClockCondition} \index{19}
	
	\begin{longdescription}
		\item[Overview] 		
				

	

		This specifies a time dependency to an events last occurence.
It leads to an introduction of an auxiliary clock.		
		\item[ESuper Types of \texttt{AuxiliaryClockCondition}] ~
			\begin{longdescription}
				\item[\texttt{ClockCondition}] see Section~\ref{cls:dependencylanguage::ClockCondition} on Page~\pageref{cls:dependencylanguage::ClockCondition}						\end{longdescription}
		
	
			\item[\textbf{EAttributes of} \texttt{AuxiliaryClockCondition}] ~
			\begin{longdescription}
	\item[\texttt{operator : ComparingOperator \symbol{"5B}1..1\symbol{"5D}
}] ~
	see Section~\ref{cls:core::expressions::common::ComparingOperator} on Page~\pageref{cls:core::expressions::common::ComparingOperator}
	
	\nopagebreak
		
				

	

	
			\end{longdescription}
			\item[\textbf{EReferences of} \texttt{AuxiliaryClockCondition}] ~
			\begin{longdescription}
	\item[\texttt{bound : TimeValue \symbol{"5B}1..1\symbol{"5D}
}] ~
	see Section~\ref{cls:muml::valuetype::TimeValue} on Page~\pageref{cls:muml::valuetype::TimeValue}
	
	\nopagebreak
		
				

	

	
	\item[\texttt{event : Event \symbol{"5B}1..1\symbol{"5D}
}] ~
	see Section~\ref{cls:dependencylanguage::Event} on Page~\pageref{cls:dependencylanguage::Event}
	
	\nopagebreak
		
				

	

	
			\end{longdescription}
	
	\end{longdescription}
	

%%%%%%%%%%%%%%%%%%%%%%%%%%%%%%
%%%%%%%%%%%%%%%%%%%%%%%%%%%%%%
%%%%%%%%%%%%%%%%%%%%%%%%%%%%%%
\subsection{EClass \bfseries \texttt{BasicClockCondition}\normalfont}
\label{cls:dependencylanguage::BasicClockCondition} \index{21}
	
	\begin{longdescription}
		\item[Overview] 		
				

	

		It is just a ClockConstraint with the additional of ClockCondition, to better integratable into the DependencyLanguage.		
		\item[ESuper Types of \texttt{BasicClockCondition}] ~
			\begin{longdescription}
				\item[\texttt{ClockConstraint}] see Section~\ref{cls:muml::realtimestatechart::ClockConstraint} on Page~\pageref{cls:muml::realtimestatechart::ClockConstraint}							\item[\texttt{ClockCondition}] see Section~\ref{cls:dependencylanguage::ClockCondition} on Page~\pageref{cls:dependencylanguage::ClockCondition}						\end{longdescription}
		
	
	
	\end{longdescription}
	

%%%%%%%%%%%%%%%%%%%%%%%%%%%%%%
%%%%%%%%%%%%%%%%%%%%%%%%%%%%%%
%%%%%%%%%%%%%%%%%%%%%%%%%%%%%%
\subsection{EClass \bfseries \texttt{BoundedActiveState}\normalfont}
\label{cls:dependencylanguage::BoundedActiveState} \index{13}
	
	\begin{longdescription}
		\item[Overview] 		
				

	

		It can specify invariants for states. Those invariants must include references of elements from other regions.
	
		\item[ESuper Types of \texttt{BoundedActiveState}] ~
			\begin{longdescription}
				\item[\texttt{Dependency}] see Section~\ref{cls:dependencylanguage::Dependency} on Page~\pageref{cls:dependencylanguage::Dependency}						\end{longdescription}
		
	
			\item[\textbf{EReferences of} \texttt{BoundedActiveState}] ~
			\begin{longdescription}
	\item[\texttt{constraint : ClockCondition \symbol{"5B}1..1\symbol{"5D}
}] ~
	see Section~\ref{cls:dependencylanguage::ClockCondition} on Page~\pageref{cls:dependencylanguage::ClockCondition}
	
	\nopagebreak
		
				

	

		It is currently only allowed to specify UpperBoundClockConditions, because invariants only support UpperBoundClockConstraints.
However, it might be possible to drop the restriction, because also other Conditions might be simulatable through UpperBoundClockConstraints.		
	\item[\texttt{states : State \symbol{"5B}1..$*$\symbol{"5D}
}] ~
	see Section~\ref{cls:muml::realtimestatechart::State} on Page~\pageref{cls:muml::realtimestatechart::State}
	
	\nopagebreak
		
				

	

	
			\end{longdescription}
	
	\end{longdescription}
	

%%%%%%%%%%%%%%%%%%%%%%%%%%%%%%
%%%%%%%%%%%%%%%%%%%%%%%%%%%%%%
%%%%%%%%%%%%%%%%%%%%%%%%%%%%%%
\subsection{EClass \bfseries \texttt{ClockCondition}\normalfont}
\label{cls:dependencylanguage::ClockCondition} \index{18}
	
	\begin{longdescription}
		\item[Overview] 		
				

	

		A clock condition describes a ClockConstraint that can be either be resolved to an existing or an auxiliary clock.		
		\item[ESuper Types of \texttt{ClockCondition}] ~
			\begin{longdescription}
				\item[\texttt{Condition}] see Section~\ref{cls:dependencylanguage::Condition} on Page~\pageref{cls:dependencylanguage::Condition}						\end{longdescription}
		
	
	
	\end{longdescription}
	

%%%%%%%%%%%%%%%%%%%%%%%%%%%%%%
%%%%%%%%%%%%%%%%%%%%%%%%%%%%%%
%%%%%%%%%%%%%%%%%%%%%%%%%%%%%%
\subsection{EClass \bfseries \texttt{ClockMerge}\normalfont}
\label{cls:dependencylanguage::ClockMerge} \index{5}
	
	\begin{longdescription}
		\item[Overview] 		
				

	

		It merges multiple clocks into one global.		
		\item[ESuper Types of \texttt{ClockMerge}] ~
			\begin{longdescription}
				\item[\texttt{Dependency}] see Section~\ref{cls:dependencylanguage::Dependency} on Page~\pageref{cls:dependencylanguage::Dependency}						\end{longdescription}
		
	
			\item[\textbf{EAttributes of} \texttt{ClockMerge}] ~
			\begin{longdescription}
	\item[\texttt{clockName : EString \symbol{"5B}0..1\symbol{"5D}
}] ~
	
	
	\nopagebreak
		
				

	

		If not specified the name of the first clock will be used for the global clock.		
			\end{longdescription}
			\item[\textbf{EReferences of} \texttt{ClockMerge}] ~
			\begin{longdescription}
	\item[\texttt{clocks : Clock \symbol{"5B}2..$*$\symbol{"5D}
}] ~
	see Section~\ref{cls:muml::realtimestatechart::Clock} on Page~\pageref{cls:muml::realtimestatechart::Clock}
	
	\nopagebreak
		
				

	

		Clocks from different regions which are merged.		
			\end{longdescription}
	
	\end{longdescription}
	

%%%%%%%%%%%%%%%%%%%%%%%%%%%%%%
%%%%%%%%%%%%%%%%%%%%%%%%%%%%%%
%%%%%%%%%%%%%%%%%%%%%%%%%%%%%%
\subsection{EClass \bfseries \texttt{ClockResetEffect}\normalfont}
\label{cls:dependencylanguage::ClockResetEffect} \index{10}
	
	\begin{longdescription}
		\item[Overview] 		
				

	

		Specifies which Clocks should be reset, when the Condition or Event happens.		
		\item[ESuper Types of \texttt{ClockResetEffect}] ~
			\begin{longdescription}
				\item[\texttt{Effect}] see Section~\ref{cls:dependencylanguage::Effect} on Page~\pageref{cls:dependencylanguage::Effect}						\end{longdescription}
		
	
			\item[\textbf{EReferences of} \texttt{ClockResetEffect}] ~
			\begin{longdescription}
	\item[\texttt{clocks : Clock \symbol{"5B}1..$*$\symbol{"5D}
}] ~
	see Section~\ref{cls:muml::realtimestatechart::Clock} on Page~\pageref{cls:muml::realtimestatechart::Clock}
	
	\nopagebreak
		
				

	

		Clocks that shall be reset.		
			\end{longdescription}
	
	\end{longdescription}
	

%%%%%%%%%%%%%%%%%%%%%%%%%%%%%%
%%%%%%%%%%%%%%%%%%%%%%%%%%%%%%
%%%%%%%%%%%%%%%%%%%%%%%%%%%%%%
\subsection{EClass \bfseries \texttt{CompositionCondition}\normalfont}
\label{cls:dependencylanguage::CompositionCondition} \index{23}
	
	\begin{longdescription}
		\item[Overview] 		
				

	

		This can be used to logically compose multiple conditions.
It is mostly required to shorten specifications, but also the OR composition is otherwise not possible.		
		\item[ESuper Types of \texttt{CompositionCondition}] ~
			\begin{longdescription}
				\item[\texttt{Condition}] see Section~\ref{cls:dependencylanguage::Condition} on Page~\pageref{cls:dependencylanguage::Condition}						\end{longdescription}
		
	
			\item[\textbf{EAttributes of} \texttt{CompositionCondition}] ~
			\begin{longdescription}
	\item[\texttt{kind : LogicOperator \symbol{"5B}1..1\symbol{"5D}
}] ~
	see Section~\ref{cls:core::expressions::common::LogicOperator} on Page~\pageref{cls:core::expressions::common::LogicOperator}
	
	\nopagebreak
		
				

	

	
			\end{longdescription}
			\item[\textbf{EReferences of} \texttt{CompositionCondition}] ~
			\begin{longdescription}
	\item[\texttt{leftCondition : Condition \symbol{"5B}0..1\symbol{"5D}
}] ~
	see Section~\ref{cls:dependencylanguage::Condition} on Page~\pageref{cls:dependencylanguage::Condition}
	
	\nopagebreak
		
				

	

	
	\item[\texttt{rightCondition : Condition \symbol{"5B}0..1\symbol{"5D}
}] ~
	see Section~\ref{cls:dependencylanguage::Condition} on Page~\pageref{cls:dependencylanguage::Condition}
	
	\nopagebreak
		
				

	

	
			\end{longdescription}
	
	\end{longdescription}
	

%%%%%%%%%%%%%%%%%%%%%%%%%%%%%%
%%%%%%%%%%%%%%%%%%%%%%%%%%%%%%
%%%%%%%%%%%%%%%%%%%%%%%%%%%%%%
\subsection{EClass \bfseries \texttt{CompositionEvent}\normalfont}
\label{cls:dependencylanguage::CompositionEvent} \index{26}
	
	\begin{longdescription}
		\item[Overview] 		
				

	

		This can be used to logically compose multiple events.
If an And composition is chosen it could require to introduce auxiliary elements.		
		\item[ESuper Types of \texttt{CompositionEvent}] ~
			\begin{longdescription}
				\item[\texttt{Event}] see Section~\ref{cls:dependencylanguage::Event} on Page~\pageref{cls:dependencylanguage::Event}						\end{longdescription}
		
	
			\item[\textbf{EAttributes of} \texttt{CompositionEvent}] ~
			\begin{longdescription}
	\item[\texttt{kind : LogicOperator \symbol{"5B}1..1\symbol{"5D}
}] ~
	see Section~\ref{cls:core::expressions::common::LogicOperator} on Page~\pageref{cls:core::expressions::common::LogicOperator}
	
	\nopagebreak
		
				

	

	
			\end{longdescription}
			\item[\textbf{EReferences of} \texttt{CompositionEvent}] ~
			\begin{longdescription}
	\item[\texttt{leftEvent : Event \symbol{"5B}1..1\symbol{"5D}
}] ~
	see Section~\ref{cls:dependencylanguage::Event} on Page~\pageref{cls:dependencylanguage::Event}
	
	\nopagebreak
		
				

	

	
	\item[\texttt{rightEvent : Event \symbol{"5B}1..1\symbol{"5D}
}] ~
	see Section~\ref{cls:dependencylanguage::Event} on Page~\pageref{cls:dependencylanguage::Event}
	
	\nopagebreak
		
				

	

	
			\end{longdescription}
	
	\end{longdescription}
	

%%%%%%%%%%%%%%%%%%%%%%%%%%%%%%
%%%%%%%%%%%%%%%%%%%%%%%%%%%%%%
%%%%%%%%%%%%%%%%%%%%%%%%%%%%%%
\subsection{Abstract EClass \bfseries \texttt{Condition}\normalfont}
\label{cls:dependencylanguage::Condition} \index{14}
	
	\begin{longdescription}
		\item[Overview] 		
				

	

		Those Conditions can evaluate to true or false and thereby specify intervals.		
		\item[ESuper Types of \texttt{Condition}] ~
			\begin{longdescription}
				\item[\texttt{EObject}] 						\end{longdescription}
		
	
	
	\end{longdescription}
	

%%%%%%%%%%%%%%%%%%%%%%%%%%%%%%
%%%%%%%%%%%%%%%%%%%%%%%%%%%%%%
%%%%%%%%%%%%%%%%%%%%%%%%%%%%%%
\subsection{EClass \bfseries \texttt{ConditionalDependency}\normalfont}
\label{cls:dependencylanguage::ConditionalDependency} \index{7}
	
	\begin{longdescription}
		\item[Overview] 		
				

	

		A conditional dependency can be used to restrict Transition by Constraints, which have to be specified as Conditions.
Through enable and disable the resolved Constraint will be attached to transitions.
Also clock resets and data assignments can be specified.
Those are applied for the specified event or when the Condition changes from false to true.		
		\item[ESuper Types of \texttt{ConditionalDependency}] ~
			\begin{longdescription}
				\item[\texttt{Dependency}] see Section~\ref{cls:dependencylanguage::Dependency} on Page~\pageref{cls:dependencylanguage::Dependency}						\end{longdescription}
		
	
			\item[\textbf{EReferences of} \texttt{ConditionalDependency}] ~
			\begin{longdescription}
	\item[\texttt{condition : Condition \symbol{"5B}0..1\symbol{"5D}
}] ~
	see Section~\ref{cls:dependencylanguage::Condition} on Page~\pageref{cls:dependencylanguage::Condition}
	
	\nopagebreak
		
				

	

		see Condition		
	\item[\texttt{effects : Effect \symbol{"5B}0..$*$\symbol{"5D}
}] ~
	see Section~\ref{cls:dependencylanguage::Effect} on Page~\pageref{cls:dependencylanguage::Effect}
	
	\nopagebreak
		
				

	

		see Effect		
	\item[\texttt{event : Event \symbol{"5B}0..1\symbol{"5D}
}] ~
	see Section~\ref{cls:dependencylanguage::Event} on Page~\pageref{cls:dependencylanguage::Event}
	
	\nopagebreak
		
				

	

		Either event or condition can be used.
Event cannot be combined with EnableDisableEffect.		
			\end{longdescription}
	
	\end{longdescription}
	

%%%%%%%%%%%%%%%%%%%%%%%%%%%%%%
%%%%%%%%%%%%%%%%%%%%%%%%%%%%%%
%%%%%%%%%%%%%%%%%%%%%%%%%%%%%%
\subsection{EClass \bfseries \texttt{CountedEvent}\normalfont}
\label{cls:dependencylanguage::CountedEvent} \index{27}
	
	\begin{longdescription}
		\item[Overview] 		
				

	

		This event counts the occurences of its subevent. Its own occurence is triggered when the threshold defined in counter is reached.		
		\item[ESuper Types of \texttt{CountedEvent}] ~
			\begin{longdescription}
				\item[\texttt{Event}] see Section~\ref{cls:dependencylanguage::Event} on Page~\pageref{cls:dependencylanguage::Event}						\end{longdescription}
		
	
			\item[\textbf{EAttributes of} \texttt{CountedEvent}] ~
			\begin{longdescription}
	\item[\texttt{counter : EInt \symbol{"5B}0..1\symbol{"5D}
}] ~
	
	
	\nopagebreak
		
				

	

	
			\end{longdescription}
			\item[\textbf{EReferences of} \texttt{CountedEvent}] ~
			\begin{longdescription}
	\item[\texttt{event : Event \symbol{"5B}1..1\symbol{"5D}
}] ~
	see Section~\ref{cls:dependencylanguage::Event} on Page~\pageref{cls:dependencylanguage::Event}
	
	\nopagebreak
		
				

	

	
			\end{longdescription}
	
	\end{longdescription}
	

%%%%%%%%%%%%%%%%%%%%%%%%%%%%%%
%%%%%%%%%%%%%%%%%%%%%%%%%%%%%%
%%%%%%%%%%%%%%%%%%%%%%%%%%%%%%
\subsection{EClass \bfseries \texttt{DataAssignmentEffect}\normalfont}
\label{cls:dependencylanguage::DataAssignmentEffect} \index{11}
	
	\begin{longdescription}
		\item[Overview] 		
				

	

		Specifies which Assignments that should be applied, when the Condition or Event happens.		
		\item[ESuper Types of \texttt{DataAssignmentEffect}] ~
			\begin{longdescription}
				\item[\texttt{Effect}] see Section~\ref{cls:dependencylanguage::Effect} on Page~\pageref{cls:dependencylanguage::Effect}							\item[\texttt{Assignment}] see Section~\ref{cls:actionlanguage::Assignment} on Page~\pageref{cls:actionlanguage::Assignment}						\end{longdescription}
		
	
			\item[\textbf{EAttributes of} \texttt{DataAssignmentEffect}] ~
			\begin{longdescription}
	\item[\texttt{/direction : AssignmentDirection \symbol{"5B}1..1\symbol{"5D}
}] ~
	see Section~\ref{cls:dependencylanguage::AssignmentDirection} on Page~\pageref{cls:dependencylanguage::AssignmentDirection}
	
	\nopagebreak
		
				

	

		Direction of the dependency (see AssignmentDirection).		
			\end{longdescription}
	
	\end{longdescription}
	

%%%%%%%%%%%%%%%%%%%%%%%%%%%%%%
%%%%%%%%%%%%%%%%%%%%%%%%%%%%%%
%%%%%%%%%%%%%%%%%%%%%%%%%%%%%%
\subsection{EClass \bfseries \texttt{DataCondition}\normalfont}
\label{cls:dependencylanguage::DataCondition} \index{22}
	
	\begin{longdescription}
		\item[Overview] 		
				

	

		This Condition can contain any Expression which can be specified for TypedNamedElements.
This is also just required for an easier integration of Expression.		
		\item[ESuper Types of \texttt{DataCondition}] ~
			\begin{longdescription}
				\item[\texttt{Condition}] see Section~\ref{cls:dependencylanguage::Condition} on Page~\pageref{cls:dependencylanguage::Condition}						\end{longdescription}
		
	
			\item[\textbf{EReferences of} \texttt{DataCondition}] ~
			\begin{longdescription}
	\item[\texttt{expression : Expression \symbol{"5B}1..1\symbol{"5D}
}] ~
	see Section~\ref{cls:core::expressions::Expression} on Page~\pageref{cls:core::expressions::Expression}
	
	\nopagebreak
		
				

	

	
			\end{longdescription}
	
	\end{longdescription}
	

%%%%%%%%%%%%%%%%%%%%%%%%%%%%%%
%%%%%%%%%%%%%%%%%%%%%%%%%%%%%%
%%%%%%%%%%%%%%%%%%%%%%%%%%%%%%
\subsection{EClass \bfseries \texttt{DataMerge}\normalfont}
\label{cls:dependencylanguage::DataMerge} \index{6}
	
	\begin{longdescription}
		\item[Overview] 		
				

	

		It merges multiple variable into one global.		
		\item[ESuper Types of \texttt{DataMerge}] ~
			\begin{longdescription}
				\item[\texttt{Dependency}] see Section~\ref{cls:dependencylanguage::Dependency} on Page~\pageref{cls:dependencylanguage::Dependency}						\end{longdescription}
		
	
			\item[\textbf{EAttributes of} \texttt{DataMerge}] ~
			\begin{longdescription}
	\item[\texttt{variableName : EString \symbol{"5B}0..1\symbol{"5D}
}] ~
	
	
	\nopagebreak
		
				

	

		If not specified the name of the first variable will be used for the global variable.		
			\end{longdescription}
			\item[\textbf{EReferences of} \texttt{DataMerge}] ~
			\begin{longdescription}
	\item[\texttt{variables : Variable \symbol{"5B}2..$*$\symbol{"5D}
}] ~
	see Section~\ref{cls:muml::behavior::Variable} on Page~\pageref{cls:muml::behavior::Variable}
	
	\nopagebreak
		
				

	

		 Variables from different regions which are merged.		
			\end{longdescription}
	
	\end{longdescription}
	

%%%%%%%%%%%%%%%%%%%%%%%%%%%%%%
%%%%%%%%%%%%%%%%%%%%%%%%%%%%%%
%%%%%%%%%%%%%%%%%%%%%%%%%%%%%%
\subsection{EClass \bfseries \texttt{DelayedEvent}\normalfont}
\label{cls:dependencylanguage::DelayedEvent} \index{28}
	
	\begin{longdescription}
		\item[Overview] 		
				

	

		This event occurs with the specified delay after its subevent occured.
It requires an auxiliary elements such that event can happen without any other dependencies.		
		\item[ESuper Types of \texttt{DelayedEvent}] ~
			\begin{longdescription}
				\item[\texttt{Event}] see Section~\ref{cls:dependencylanguage::Event} on Page~\pageref{cls:dependencylanguage::Event}						\end{longdescription}
		
	
			\item[\textbf{EReferences of} \texttt{DelayedEvent}] ~
			\begin{longdescription}
	\item[\texttt{delay : TimeValue \symbol{"5B}1..1\symbol{"5D}
}] ~
	see Section~\ref{cls:muml::valuetype::TimeValue} on Page~\pageref{cls:muml::valuetype::TimeValue}
	
	\nopagebreak
		
				

	

	
	\item[\texttt{event : Event \symbol{"5B}1..1\symbol{"5D}
}] ~
	see Section~\ref{cls:dependencylanguage::Event} on Page~\pageref{cls:dependencylanguage::Event}
	
	\nopagebreak
		
				

	

	
			\end{longdescription}
	
	\end{longdescription}
	

%%%%%%%%%%%%%%%%%%%%%%%%%%%%%%
%%%%%%%%%%%%%%%%%%%%%%%%%%%%%%
%%%%%%%%%%%%%%%%%%%%%%%%%%%%%%
\subsection{Abstract EClass \bfseries \texttt{Dependency}\normalfont}
\label{cls:dependencylanguage::Dependency} \index{2}
	
	\begin{longdescription}
		\item[Overview] 		
				

	

		Each dependency must at least contain two references that lead to two different RealtimeStatecharts.
This could be check by constraints.		
		\item[ESuper Types of \texttt{Dependency}] ~
			\begin{longdescription}
				\item[\texttt{CommentableElement}] see Section~\ref{cls:core::CommentableElement} on Page~\pageref{cls:core::CommentableElement}						\end{longdescription}
		
	
	
	\end{longdescription}
	

%%%%%%%%%%%%%%%%%%%%%%%%%%%%%%
%%%%%%%%%%%%%%%%%%%%%%%%%%%%%%
%%%%%%%%%%%%%%%%%%%%%%%%%%%%%%
\subsection{EClass \bfseries \texttt{DependencyModel}\normalfont}
\label{cls:dependencylanguage::DependencyModel} \index{1}
	
	\begin{longdescription}
		\item[Overview] 		
				

	

		The dependency model is the container for all dependencies and it builds the root for the XText grammar.		
		\item[ESuper Types of \texttt{DependencyModel}] ~
			\begin{longdescription}
				\item[\texttt{CommentableElement}] see Section~\ref{cls:core::CommentableElement} on Page~\pageref{cls:core::CommentableElement}						\end{longdescription}
		
	
			\item[\textbf{EReferences of} \texttt{DependencyModel}] ~
			\begin{longdescription}
	\item[\texttt{dependencies : Dependency \symbol{"5B}0..$*$\symbol{"5D}
}] ~
	see Section~\ref{cls:dependencylanguage::Dependency} on Page~\pageref{cls:dependencylanguage::Dependency}
	
	\nopagebreak
		
				

	

		  see Dependency		
			\end{longdescription}
	
	\end{longdescription}
	

%%%%%%%%%%%%%%%%%%%%%%%%%%%%%%
%%%%%%%%%%%%%%%%%%%%%%%%%%%%%%
%%%%%%%%%%%%%%%%%%%%%%%%%%%%%%
\subsection{Abstract EClass \bfseries \texttt{Effect}\normalfont}
\label{cls:dependencylanguage::Effect} \index{8}
	
	\begin{longdescription}
		\item[Overview] 		
				

	

		The effect species for what the condition is used, e.g.,
as Expression or as Position for Assignments.		
		\item[ESuper Types of \texttt{Effect}] ~
			\begin{longdescription}
				\item[\texttt{EObject}] 						\end{longdescription}
		
	
	
	\end{longdescription}
	

%%%%%%%%%%%%%%%%%%%%%%%%%%%%%%
%%%%%%%%%%%%%%%%%%%%%%%%%%%%%%
%%%%%%%%%%%%%%%%%%%%%%%%%%%%%%
\subsection{EClass \bfseries \texttt{EnableDisableEffect}\normalfont}
\label{cls:dependencylanguage::EnableDisableEffect} \index{9}
	
	\begin{longdescription}
		\item[Overview] 		
				

	

		It restricts its event that the event can only occur if the condition is met.
	
		\item[ESuper Types of \texttt{EnableDisableEffect}] ~
			\begin{longdescription}
				\item[\texttt{Effect}] see Section~\ref{cls:dependencylanguage::Effect} on Page~\pageref{cls:dependencylanguage::Effect}						\end{longdescription}
		
	
			\item[\textbf{EAttributes of} \texttt{EnableDisableEffect}] ~
			\begin{longdescription}
	\item[\texttt{isEnable : EBoolean \symbol{"5B}0..1\symbol{"5D}
}] ~
	
	
	\nopagebreak
		
				

	

		It is used to invert the 
If it is false the Expression is invertingly applied as the restriction.		
			\end{longdescription}
			\item[\textbf{EReferences of} \texttt{EnableDisableEffect}] ~
			\begin{longdescription}
	\item[\texttt{event : Event \symbol{"5B}1..1\symbol{"5D}
}] ~
	see Section~\ref{cls:dependencylanguage::Event} on Page~\pageref{cls:dependencylanguage::Event}
	
	\nopagebreak
		
				

	

		Transition or EntryOrExitEvent that shall be restricted (see Event).		
			\end{longdescription}
	
	\end{longdescription}
	

%%%%%%%%%%%%%%%%%%%%%%%%%%%%%%
%%%%%%%%%%%%%%%%%%%%%%%%%%%%%%
%%%%%%%%%%%%%%%%%%%%%%%%%%%%%%
\subsection{Abstract EClass \bfseries \texttt{Event}\normalfont}
\label{cls:dependencylanguage::Event} \index{24}
	
	\begin{longdescription}
		\item[Overview] 		
				

	

		Events can be resolved to existing or auxiliary Transitions and EntryOrExitEvents.		
		\item[ESuper Types of \texttt{Event}] ~
			\begin{longdescription}
				\item[\texttt{EObject}] 						\end{longdescription}
		
	
	
	\end{longdescription}
	

%%%%%%%%%%%%%%%%%%%%%%%%%%%%%%
%%%%%%%%%%%%%%%%%%%%%%%%%%%%%%
%%%%%%%%%%%%%%%%%%%%%%%%%%%%%%
\subsection{EClass \bfseries \texttt{EventConstrainedIntervalCondition}\normalfont}
\label{cls:dependencylanguage::EventConstrainedIntervalCondition} \index{16}
	
	\begin{longdescription}
		\item[Overview] 		
				

	

		This condition specifies an interval in relation to two Events.
initialEnabled and enabledInfinite should not be used, because it would be the same as writing true as Condition.		
		\item[ESuper Types of \texttt{EventConstrainedIntervalCondition}] ~
			\begin{longdescription}
				\item[\texttt{Condition}] see Section~\ref{cls:dependencylanguage::Condition} on Page~\pageref{cls:dependencylanguage::Condition}						\end{longdescription}
		
	
			\item[\textbf{EAttributes of} \texttt{EventConstrainedIntervalCondition}] ~
			\begin{longdescription}
	\item[\texttt{enabledInfite : EBoolean \symbol{"5B}1..1\symbol{"5D}
}] ~
	
	
	\nopagebreak
		
				

	

		When it is true, the evaluation never changes after it is true.		
	\item[\texttt{initialEnabled : EBoolean \symbol{"5B}1..1\symbol{"5D}
}] ~
	
	
	\nopagebreak
		
				

	

		When it is true, the evaluation after initialization true.		
			\end{longdescription}
			\item[\textbf{EReferences of} \texttt{EventConstrainedIntervalCondition}] ~
			\begin{longdescription}
	\item[\texttt{fromEvent : Event \symbol{"5B}0..1\symbol{"5D}
}] ~
	see Section~\ref{cls:dependencylanguage::Event} on Page~\pageref{cls:dependencylanguage::Event}
	
	\nopagebreak
		
				

	

		When this event occurs, the condition evaluation is changed to true.		
	\item[\texttt{untilEvent : Event \symbol{"5B}0..1\symbol{"5D}
}] ~
	see Section~\ref{cls:dependencylanguage::Event} on Page~\pageref{cls:dependencylanguage::Event}
	
	\nopagebreak
		
				

	

		When this event occurs, the condition evaluation is changed to false		
			\end{longdescription}
	
	\end{longdescription}
	

%%%%%%%%%%%%%%%%%%%%%%%%%%%%%%
%%%%%%%%%%%%%%%%%%%%%%%%%%%%%%
%%%%%%%%%%%%%%%%%%%%%%%%%%%%%%
\subsection{EClass \bfseries \texttt{ForbiddenStateCombination}\normalfont}
\label{cls:dependencylanguage::ForbiddenStateCombination} \index{4}
	
	\begin{longdescription}
		\item[Overview] 		
				

	

		It forbids the combination of states. This dependency requires to add channels and an auxiliary region.		
		\item[ESuper Types of \texttt{ForbiddenStateCombination}] ~
			\begin{longdescription}
				\item[\texttt{Dependency}] see Section~\ref{cls:dependencylanguage::Dependency} on Page~\pageref{cls:dependencylanguage::Dependency}						\end{longdescription}
		
	
			\item[\textbf{EReferences of} \texttt{ForbiddenStateCombination}] ~
			\begin{longdescription}
	\item[\texttt{states : State \symbol{"5B}2..$*$\symbol{"5D}
}] ~
	see Section~\ref{cls:muml::realtimestatechart::State} on Page~\pageref{cls:muml::realtimestatechart::State}
	
	\nopagebreak
		
				

	

		Those states are not allowed to be entered in combination.
If it contains more than two states, it does not forbid each possible combination, but only the one which is specified.		
			\end{longdescription}
	
	\end{longdescription}
	

%%%%%%%%%%%%%%%%%%%%%%%%%%%%%%
%%%%%%%%%%%%%%%%%%%%%%%%%%%%%%
%%%%%%%%%%%%%%%%%%%%%%%%%%%%%%
\subsection{EClass \bfseries \texttt{HybridClockCondition}\normalfont}
\label{cls:dependencylanguage::HybridClockCondition} \index{20}
	
	\begin{longdescription}
		\item[Overview] 		
				

	

		This is a abreviation of a AuxiliaryClockCondition and a Condtion (mainly for StateStatusConditions).
For example:
HybridClockCondition(condition:= state 1 active ,  operation:= < , bound:= 5s )
<=>
AuxiliaryClockCondition(event:= entering state 1,  operation:= < , bound:= 5s ) and StateStatusCondition(states:= state 1 , kind:= active)		
		\item[ESuper Types of \texttt{HybridClockCondition}] ~
			\begin{longdescription}
				\item[\texttt{ClockCondition}] see Section~\ref{cls:dependencylanguage::ClockCondition} on Page~\pageref{cls:dependencylanguage::ClockCondition}						\end{longdescription}
		
	
			\item[\textbf{EAttributes of} \texttt{HybridClockCondition}] ~
			\begin{longdescription}
	\item[\texttt{operator : ComparingOperator \symbol{"5B}1..1\symbol{"5D}
}] ~
	see Section~\ref{cls:core::expressions::common::ComparingOperator} on Page~\pageref{cls:core::expressions::common::ComparingOperator}
	
	\nopagebreak
		
				

	

	
			\end{longdescription}
			\item[\textbf{EReferences of} \texttt{HybridClockCondition}] ~
			\begin{longdescription}
	\item[\texttt{bound : TimeValue \symbol{"5B}1..1\symbol{"5D}
}] ~
	see Section~\ref{cls:muml::valuetype::TimeValue} on Page~\pageref{cls:muml::valuetype::TimeValue}
	
	\nopagebreak
		
				

	

	
	\item[\texttt{condition : Condition \symbol{"5B}1..1\symbol{"5D}
}] ~
	see Section~\ref{cls:dependencylanguage::Condition} on Page~\pageref{cls:dependencylanguage::Condition}
	
	\nopagebreak
		
				

	

	
			\end{longdescription}
	
	\end{longdescription}
	

%%%%%%%%%%%%%%%%%%%%%%%%%%%%%%
%%%%%%%%%%%%%%%%%%%%%%%%%%%%%%
%%%%%%%%%%%%%%%%%%%%%%%%%%%%%%
\subsection{EClass \bfseries \texttt{MessageEvent}\normalfont}
\label{cls:dependencylanguage::MessageEvent} \index{35}
	
	\begin{longdescription}
		\item[Overview] 		
				

	

		Can be resolved to a Set of Transitions, that uses the declared Message.		
		\item[ESuper Types of \texttt{MessageEvent}] ~
			\begin{longdescription}
				\item[\texttt{SimpleEvent}] see Section~\ref{cls:dependencylanguage::SimpleEvent} on Page~\pageref{cls:dependencylanguage::SimpleEvent}						\end{longdescription}
		
	
			\item[\textbf{EAttributes of} \texttt{MessageEvent}] ~
			\begin{longdescription}
	\item[\texttt{kind : MessageEventKind \symbol{"5B}1..1\symbol{"5D}
}] ~
	see Section~\ref{cls:dependencylanguage::MessageEventKind} on Page~\pageref{cls:dependencylanguage::MessageEventKind}
	
	\nopagebreak
		
				

	

		 see MessageEventKind		
			\end{longdescription}
			\item[\textbf{EReferences of} \texttt{MessageEvent}] ~
			\begin{longdescription}
	\item[\texttt{port : DiscreteInteractionEndpoint \symbol{"5B}0..1\symbol{"5D}
}] ~
	see Section~\ref{cls:muml::connector::DiscreteInteractionEndpoint} on Page~\pageref{cls:muml::connector::DiscreteInteractionEndpoint}
	
	\nopagebreak
		
				

	

		The Message Event can be optionally bound to one port.		
	\item[\texttt{type : MessageType \symbol{"5B}1..1\symbol{"5D}
}] ~
	see Section~\ref{cls:muml::msgtype::MessageType} on Page~\pageref{cls:muml::msgtype::MessageType}
	
	\nopagebreak
		
				

	

		 Message which is send or consumed.		
			\end{longdescription}
	
	\end{longdescription}
	

%%%%%%%%%%%%%%%%%%%%%%%%%%%%%%
%%%%%%%%%%%%%%%%%%%%%%%%%%%%%%
%%%%%%%%%%%%%%%%%%%%%%%%%%%%%%
\subsection{EEnumeration \bfseries \texttt{MessageEventKind}\normalfont}
\label{cls:dependencylanguage::MessageEventKind} \index{dependencylanguage!MessageEventKind}

	\begin{longdescription}
		\item[Overview] 		
				

	

		A message can either be send or consumed at a Transition.		
	
		\item[\textbf{ELiterals of} \texttt{MessageEventKind}] ~
		\begin{longdescription}
			
\item[\texttt{UNDEFIEND = 0}] ~
\nopagebreak

\item[\texttt{CONSUMING = 1}] ~
\nopagebreak

\item[\texttt{SENDING = 2}] ~
\nopagebreak
		\end{longdescription}
	\end{longdescription}
	
	

%%%%%%%%%%%%%%%%%%%%%%%%%%%%%%
%%%%%%%%%%%%%%%%%%%%%%%%%%%%%%
%%%%%%%%%%%%%%%%%%%%%%%%%%%%%%
\subsection{Abstract EClass \bfseries \texttt{SimpleEvent}\normalfont}
\label{cls:dependencylanguage::SimpleEvent} \index{25}
	
	\begin{longdescription}
		\item[Overview] 		
				

	

		They should be always resolveable to existing elements.		
		\item[ESuper Types of \texttt{SimpleEvent}] ~
			\begin{longdescription}
				\item[\texttt{Event}] see Section~\ref{cls:dependencylanguage::Event} on Page~\pageref{cls:dependencylanguage::Event}						\end{longdescription}
		
	
	
	\end{longdescription}
	

%%%%%%%%%%%%%%%%%%%%%%%%%%%%%%
%%%%%%%%%%%%%%%%%%%%%%%%%%%%%%
%%%%%%%%%%%%%%%%%%%%%%%%%%%%%%
\subsection{EClass \bfseries \texttt{StateCombinationEvent}\normalfont}
\label{cls:dependencylanguage::StateCombinationEvent} \index{33}
	
	\begin{longdescription}
		\item[Overview] 		
				

	

		Can be resolved either to a Set of Transitions or to a Set of EntryOrExitEvents.		
		\item[ESuper Types of \texttt{StateCombinationEvent}] ~
			\begin{longdescription}
				\item[\texttt{SimpleEvent}] see Section~\ref{cls:dependencylanguage::SimpleEvent} on Page~\pageref{cls:dependencylanguage::SimpleEvent}						\end{longdescription}
		
	
			\item[\textbf{EAttributes of} \texttt{StateCombinationEvent}] ~
			\begin{longdescription}
	\item[\texttt{kind : StateEventKind \symbol{"5B}1..1\symbol{"5D}
}] ~
	see Section~\ref{cls:dependencylanguage::StateEventKind} on Page~\pageref{cls:dependencylanguage::StateEventKind}
	
	\nopagebreak
		
				

	

		see StateEventKind		
			\end{longdescription}
			\item[\textbf{EReferences of} \texttt{StateCombinationEvent}] ~
			\begin{longdescription}
	\item[\texttt{states : State \symbol{"5B}2..$*$\symbol{"5D}
}] ~
	see Section~\ref{cls:muml::realtimestatechart::State} on Page~\pageref{cls:muml::realtimestatechart::State}
	
	\nopagebreak
		
				

	

	
			\end{longdescription}
	
	\end{longdescription}
	

%%%%%%%%%%%%%%%%%%%%%%%%%%%%%%
%%%%%%%%%%%%%%%%%%%%%%%%%%%%%%
%%%%%%%%%%%%%%%%%%%%%%%%%%%%%%
\subsection{EClass \bfseries \texttt{StateEvent}\normalfont}
\label{cls:dependencylanguage::StateEvent} \index{32}
	
	\begin{longdescription}
		\item[Overview] 		
				

	

		Can be resolved either to a Set of Transitions or to an EntryOrExitEvent.		
		\item[ESuper Types of \texttt{StateEvent}] ~
			\begin{longdescription}
				\item[\texttt{SimpleEvent}] see Section~\ref{cls:dependencylanguage::SimpleEvent} on Page~\pageref{cls:dependencylanguage::SimpleEvent}						\end{longdescription}
		
	
			\item[\textbf{EAttributes of} \texttt{StateEvent}] ~
			\begin{longdescription}
	\item[\texttt{kind : StateEventKind \symbol{"5B}1..1\symbol{"5D}
}] ~
	see Section~\ref{cls:dependencylanguage::StateEventKind} on Page~\pageref{cls:dependencylanguage::StateEventKind}
	
	\nopagebreak
		
				

	

		see StateEventKind		
			\end{longdescription}
			\item[\textbf{EReferences of} \texttt{StateEvent}] ~
			\begin{longdescription}
	\item[\texttt{state : State \symbol{"5B}1..1\symbol{"5D}
}] ~
	see Section~\ref{cls:muml::realtimestatechart::State} on Page~\pageref{cls:muml::realtimestatechart::State}
	
	\nopagebreak
		
				

	

	
			\end{longdescription}
	
	\end{longdescription}
	

%%%%%%%%%%%%%%%%%%%%%%%%%%%%%%
%%%%%%%%%%%%%%%%%%%%%%%%%%%%%%
%%%%%%%%%%%%%%%%%%%%%%%%%%%%%%
\subsection{EEnumeration \bfseries \texttt{StateEventKind}\normalfont}
\label{cls:dependencylanguage::StateEventKind} \index{dependencylanguage!StateEventKind}

	\begin{longdescription}
		\item[Overview] 		
				

	

		This is used to reference the EntryEvent and ExitEvent of a state.
Even it is was not instantiate yet.		
	
		\item[\textbf{ELiterals of} \texttt{StateEventKind}] ~
		\begin{longdescription}
			
\item[\texttt{UNDEFIEND = 0}] ~
\nopagebreak

\item[\texttt{ENTRY = 1}] ~
\nopagebreak

\item[\texttt{EXIT = 2}] ~
\nopagebreak
		\end{longdescription}
	\end{longdescription}
	
	

%%%%%%%%%%%%%%%%%%%%%%%%%%%%%%
%%%%%%%%%%%%%%%%%%%%%%%%%%%%%%
%%%%%%%%%%%%%%%%%%%%%%%%%%%%%%
\subsection{EClass \bfseries \texttt{StateStatusCondition}\normalfont}
\label{cls:dependencylanguage::StateStatusCondition} \index{15}
	
	\begin{longdescription}
		\item[Overview] 		
				

	

		This condition specifies the interval that a state (combination) is active or inactive.		
		\item[ESuper Types of \texttt{StateStatusCondition}] ~
			\begin{longdescription}
				\item[\texttt{Condition}] see Section~\ref{cls:dependencylanguage::Condition} on Page~\pageref{cls:dependencylanguage::Condition}						\end{longdescription}
		
	
			\item[\textbf{EAttributes of} \texttt{StateStatusCondition}] ~
			\begin{longdescription}
	\item[\texttt{kind : StateStatusKind \symbol{"5B}1..1\symbol{"5D}
}] ~
	see Section~\ref{cls:dependencylanguage::StateStatusKind} on Page~\pageref{cls:dependencylanguage::StateStatusKind}
	
	\nopagebreak
		
				

	

		see StateStatusKind		
			\end{longdescription}
			\item[\textbf{EReferences of} \texttt{StateStatusCondition}] ~
			\begin{longdescription}
	\item[\texttt{states : State \symbol{"5B}1..$*$\symbol{"5D}
}] ~
	see Section~\ref{cls:muml::realtimestatechart::State} on Page~\pageref{cls:muml::realtimestatechart::State}
	
	\nopagebreak
		
				

	

	
			\end{longdescription}
	
	\end{longdescription}
	

%%%%%%%%%%%%%%%%%%%%%%%%%%%%%%
%%%%%%%%%%%%%%%%%%%%%%%%%%%%%%
%%%%%%%%%%%%%%%%%%%%%%%%%%%%%%
\subsection{EEnumeration \bfseries \texttt{StateStatusKind}\normalfont}
\label{cls:dependencylanguage::StateStatusKind} \index{dependencylanguage!StateStatusKind}

	\begin{longdescription}
		\item[Overview] 		
				

	

		It is used to specify that a state (combination) is active or inactive.		
	
		\item[\textbf{ELiterals of} \texttt{StateStatusKind}] ~
		\begin{longdescription}
			
\item[\texttt{UNDEFIEND = 0}] ~
\nopagebreak

\item[\texttt{ACTIVE = 1}] ~
\nopagebreak

\item[\texttt{INACTIVE = 2}] ~
\nopagebreak
		\end{longdescription}
	\end{longdescription}
	
	

%%%%%%%%%%%%%%%%%%%%%%%%%%%%%%
%%%%%%%%%%%%%%%%%%%%%%%%%%%%%%
%%%%%%%%%%%%%%%%%%%%%%%%%%%%%%
\subsection{EClass \bfseries \texttt{Synchronization}\normalfont}
\label{cls:dependencylanguage::Synchronization} \index{3}
	
	\begin{longdescription}
		\item[Overview] 		
				

	

		The Synchronization describes which Transitions should be coupled.
Each Transition can only be referenced once in this expression.
This could be check by constraints.		
		\item[ESuper Types of \texttt{Synchronization}] ~
			\begin{longdescription}
				\item[\texttt{Dependency}] see Section~\ref{cls:dependencylanguage::Dependency} on Page~\pageref{cls:dependencylanguage::Dependency}						\end{longdescription}
		
	
			\item[\textbf{EAttributes of} \texttt{Synchronization}] ~
			\begin{longdescription}
	\item[\texttt{channelName : EString \symbol{"5B}0..1\symbol{"5D}
}] ~
	
	
	\nopagebreak
		
				

	

		Name for a new SynchronizationChannel. If channel with same name already exists, it is used, instead of a new created SynchronizationChannel.
If it is empty an unique name is generated.		
			\end{longdescription}
			\item[\textbf{EReferences of} \texttt{Synchronization}] ~
			\begin{longdescription}
	\item[\texttt{generalSelectorExpression : Expression \symbol{"5B}0..1\symbol{"5D}
}] ~
	see Section~\ref{cls:core::expressions::Expression} on Page~\pageref{cls:core::expressions::Expression}
	
	\nopagebreak
		
				

	

		This expression is used as default selector expression.
Thereby, it can be used either SynchronizationEvent or any other SimpleEvent.		
	\item[\texttt{receivingEvents : Event \symbol{"5B}0..$*$\symbol{"5D}
}] ~
	see Section~\ref{cls:dependencylanguage::Event} on Page~\pageref{cls:dependencylanguage::Event}
	
	\nopagebreak
		
				

	

		They can be  SimpleEvents or  SynchronizationEvents.
The second can only be used when a selectorType is defined.		
	\item[\texttt{selectorType : DataType \symbol{"5B}0..1\symbol{"5D}
}] ~
	see Section~\ref{cls:muml::types::DataType} on Page~\pageref{cls:muml::types::DataType}
	
	\nopagebreak
		
				

	

		Is optional and allows to specify a channel which uses the chosen selectortype.
If it is empty it is a simple channel.		
	\item[\texttt{sendingEvents : Event \symbol{"5B}0..$*$\symbol{"5D}
}] ~
	see Section~\ref{cls:dependencylanguage::Event} on Page~\pageref{cls:dependencylanguage::Event}
	
	\nopagebreak
		
				

	

		They can be  SimpleEvents or  SynchronizationEvents.
The second can only be used when a selectorType is defined.		
			\end{longdescription}
	
	\end{longdescription}
	

%%%%%%%%%%%%%%%%%%%%%%%%%%%%%%
%%%%%%%%%%%%%%%%%%%%%%%%%%%%%%
%%%%%%%%%%%%%%%%%%%%%%%%%%%%%%
\subsection{EClass \bfseries \texttt{SynchronizationEvent}\normalfont}
\label{cls:dependencylanguage::SynchronizationEvent} \index{29}
	
	\begin{longdescription}
		\item[Overview] 		
				

	

		It is required to specify selector expressions for single events that are referenced in Synchronization.
It is only allowed to be used if a selector type is specified.
	
		\item[ESuper Types of \texttt{SynchronizationEvent}] ~
			\begin{longdescription}
				\item[\texttt{Event}] see Section~\ref{cls:dependencylanguage::Event} on Page~\pageref{cls:dependencylanguage::Event}						\end{longdescription}
		
	
			\item[\textbf{EReferences of} \texttt{SynchronizationEvent}] ~
			\begin{longdescription}
	\item[\texttt{event : Event \symbol{"5B}1..1\symbol{"5D}
}] ~
	see Section~\ref{cls:dependencylanguage::Event} on Page~\pageref{cls:dependencylanguage::Event}
	
	\nopagebreak
		
				

	

	
	\item[\texttt{selectorExpression : Expression \symbol{"5B}0..1\symbol{"5D}
}] ~
	see Section~\ref{cls:core::expressions::Expression} on Page~\pageref{cls:core::expressions::Expression}
	
	\nopagebreak
		
				

	

	
			\end{longdescription}
	
	\end{longdescription}
	

%%%%%%%%%%%%%%%%%%%%%%%%%%%%%%
%%%%%%%%%%%%%%%%%%%%%%%%%%%%%%
%%%%%%%%%%%%%%%%%%%%%%%%%%%%%%
\subsection{EClass \bfseries \texttt{SynthesizableBehavior}\normalfont}
\label{cls:dependencylanguage::SynthesizableBehavior} \index{0}
	
	\begin{longdescription}
		\item[Overview] 		
				

	

		This class is the container for the dependencies.
It inherits from BehavioralElement to store the inner-behavior.
It can be stored at a AtomicComponent through the Extension inheritance.		
		\item[ESuper Types of \texttt{SynthesizableBehavior}] ~
			\begin{longdescription}
				\item[\texttt{Extension}] see Section~\ref{cls:core::Extension} on Page~\pageref{cls:core::Extension}							\item[\texttt{BehavioralElement}] see Section~\ref{cls:muml::behavior::BehavioralElement} on Page~\pageref{cls:muml::behavior::BehavioralElement}						\end{longdescription}
		
	
			\item[\textbf{EAttributes of} \texttt{SynthesizableBehavior}] ~
			\begin{longdescription}
	\item[\texttt{/name : EString \symbol{"5B}0..1\symbol{"5D}
}] ~
	
	
	\nopagebreak
		
				

	

		 Name of the Atomic Component + '-synth'		
		\begin{longdescription}
	\item[\small\textit{derivation}] ~ 
	\nopagebreak
		\begin{lstlisting}[language=OCL, breaklines=true]
if(not self.base.oclIsUndefined() and self.base.oclIsKindOf(component::AtomicComponent))
then (self.base).oclAsType(component::AtomicComponent).name.concat('-synth')
else
null endif		\end{lstlisting}
		\end{longdescription}
			\end{longdescription}
			\item[\textbf{EReferences of} \texttt{SynthesizableBehavior}] ~
			\begin{longdescription}
	\item[\texttt{dependencyModel : DependencyModel \symbol{"5B}0..1\symbol{"5D}
}] ~
	see Section~\ref{cls:dependencylanguage::DependencyModel} on Page~\pageref{cls:dependencylanguage::DependencyModel}
	
	\nopagebreak
		
				

	

		Container for all Dependency of one Atomic Component, Input for the XText Editor.		
			\end{longdescription}
	
	\end{longdescription}
	

%%%%%%%%%%%%%%%%%%%%%%%%%%%%%%
%%%%%%%%%%%%%%%%%%%%%%%%%%%%%%
%%%%%%%%%%%%%%%%%%%%%%%%%%%%%%
\subsection{EClass \bfseries \texttt{TransitionEvent}\normalfont}
\label{cls:dependencylanguage::TransitionEvent} \index{30}
	
	\begin{longdescription}
		\item[Overview] 		
				

	

		Resolves to the single Transition it references.
It is the simplest Event.		
		\item[ESuper Types of \texttt{TransitionEvent}] ~
			\begin{longdescription}
				\item[\texttt{SimpleEvent}] see Section~\ref{cls:dependencylanguage::SimpleEvent} on Page~\pageref{cls:dependencylanguage::SimpleEvent}						\end{longdescription}
		
	
			\item[\textbf{EReferences of} \texttt{TransitionEvent}] ~
			\begin{longdescription}
	\item[\texttt{transition : Transition \symbol{"5B}1..1\symbol{"5D}
}] ~
	see Section~\ref{cls:muml::realtimestatechart::Transition} on Page~\pageref{cls:muml::realtimestatechart::Transition}
	
	\nopagebreak
		
				

	

	
			\end{longdescription}
	
	\end{longdescription}
	
			\newpage
	